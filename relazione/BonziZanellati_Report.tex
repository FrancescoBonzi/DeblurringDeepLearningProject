\documentclass[12pt,a4paper]{article}
\usepackage[T1]{fontenc}
\usepackage[utf8]{inputenc}
\usepackage[english]{babel}
\usepackage{hyperref}
\usepackage{graphicx}
\usepackage{caption}
\captionsetup{figureposition=bottom, font=small}

\title{Deblurring Images with Autoencoders Neural Networks}
\author{Francesco Bonzi and Andrea Zanellati}
\date{}

\begin{document}
\maketitle

\section*{Introduction}
Blurring is the process of altering a region of a signal generally due to  a weighted sums of neighboring regions of the same signal. In the case of image blurring, a pixel’s value is affected by the adjacent pixels. There are several causes for blurring artefacts such as\cite {P&V&G}: 
\begin{itemize}
\item \textit{motion blur} due to movement of the object or imaging system;
\item \textit{out of focus} due to incorrect focus such as long exposure time;
\item \textit{Gaussian blur}, that is the result of blurring an image by a Gaussian function, with the main purpose to reduce the details of an image.
\end{itemize}

Deblurring is the process of removing blurring artifacts from images, that is to say that given a blurred image $B$ we try to recover the sharp image $S$. The problem can be mathematically represented with the equation \begin{equation} B = S * K \end{equation} where $K$ represents the unknown linear shift-invariant Point Spread Function (PSF). This function describes the response of an imaging system to a bright pixel (point source) by expressing the degree to wich an optical sistem blurs a point of light.

PSF play a key role in standar deblurring techniques and we can distingush two different: on one hand \textit{bind deconvolution methods} focus on recovering the unknown PSF, on the other hand \textit{non-blind methods} rely on a known PSF for performing robust deconvolution. Deblurring techniques can be split into two distinct problems: recovering the PSF by solving an inverse ill-posed problem and recovering the initial estimate using a known PSF. There are several numerical techniques developed to address the problem \cite{H&Al}. In our project we deal with a deep neural network approach which can be collocated among the blind deblurring techniques where the PSF function is represented by mean of the neural network. The input of the training system is a set of blurred images and the target output is the corresponding sharp images. In a way, the training phase of the neural network can be seen as the  PSF recovering while the prediction phase is the second part of the deblurting problem. In particular we have developed three \textit{autoencoder} architectures, whose main features are highlighted in the following section. 

Two different Dataset are considered: CIFAR10 and REDs. As regards the first, it is a set of low resolution images  

\section{Architectures of the Neural Networks}
An autoencoder is an artificial neural network that learns to copy its input to its output. It is constituted by two main parts: an encoder that compress the input into a lower-dimensional code, and a decoder that maps the code to a reconstruction of the original input. The code is a compact “summary” or “compression” of the input, also called the latent-space representation. In our autoencoder architectures the main goal is not to generate an exact copy of the original input, but rather to achieve a good representation by changing the reconstruction criterion. Indeed, the networks take a partially corrupted input and are trained to recover the original undistorted input. In practice, the objective of deblurring autoencoders is that of cleaning the corrupted input, or deblurring. 

In order to address the encoding problem Convolutional Neural Network layers (CNN) are employed, while the decoding phase is faced with deconvolution layers. CNN layers with different filters have the power to extract patterns and the main features from the input images which are saved in the latent space representation. Our simplest architecture is a plain network with three convolutional layers (encoders) and three deconvolutional layers (decoder) follow by a fourth deconvolution layer which returns the output image with the number of channels equal to those of the input image, without changing the width and height.


\newpage
\begin{thebibliography}{99}
\bibitem{P&V&G}  Patel, Amit and Vankawala, Fagun and Ganatra, Amit. (2015). A Survey on Different Image Deblurring Techniques. International Journal of Computer Applications. 116. 10.5120/20396-2697. 
\bibitem{H&Al} I.M. El-Henway et Al. (2014). A comparative study on Image Deblurring Techniques. International Journal of Advances in Computer Science and Technology. Vol 3. No 12.

\end{thebibliography}

\end{document}